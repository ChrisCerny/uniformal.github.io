\documentclass{beamer}
\usepackage{graphicx,url}
\usepackage{amsmath}
\usepackage{amssymb}
\usepackage{xkeyval}

\usepackage{mytikz}

\usepackage{basics}
\usepackage{basics-slides}

\newcommand{\lf}{\mathit{LF}}
\newcommand{\kity}{\mathit{type}}

\renewcommand{\emph}[1]{\alert{#1}}

\hypersetup{colorlinks}

% building, browsing
% Mihnea can set up server for examples, tutorial as part of examples
% TODO: clean up urtheories, examples: only 2 prover examples fail (successful version is pushed)

\begin{document}

\title{MMT Tutorial, Part 2: \\ Application Development with MMT}
\author{Florian Rabe, Mihnea Iancu, Dennis M\"uller}
\institute{Jacobs University Bremen}
\date{CICM 2016}
\begin{frame}
    \titlepage 
\begin{center}
Bringing your notebook is recommended but not required.\\
Attending \hyperref{http://www.cicm-conference.org/2016/slides/MMTLanguages.pdf}{}{}{Part 1} is helpful but not required to follow Part 2.
\end{center}
\end{frame}


\section{Developing based on MMT}

\begin{myframe}{Users vs. Developers}
MMT blurs distinction between users and developers
\begin{itemize}
 \item Intended users: developers of math applications
 \item MMT is not an application itself
 \item It is an
  \begin{itemize}
    \item API for the MMT language
     \lec{close relative of OMDoc}
    \item suite of reusable components for math applications
      \lec{e.g., MKM services}
    \item set of few example applications
       \lec{e.g., the IDE used in Part 1}
   \end{itemize}
\end{itemize}
\end{myframe}

\begin{myframe}{Extension Interfaces}
MMT is highly extensible through systematically exposed extension interfaces
  \lec{essentially everything can be replaced or customized}
\begin{itemize}
 \item Interfaces for lexer, parser, checker, simplifier, prover, presenter
   \begin{itemize}
    \item extensible by adding new rules
    \item independently replacable with custom implementations
   \end{itemize}
 \item Adding new language features
 \item Import/export interfaces for integrating other formats and build targets
 \item Exposing functionality to outside
  \begin{itemize}
    \item adding new command line syntax
    \item web framework for adding new HTTP interfaces
  \end{itemize}
 \item Change listening infrastructure for content events
\end{itemize}
\end{myframe}


\section{Structure of the Tutorial}

\begin{myframe}{Overview}
\begin{enumerate}
 \item Brief introduction to MMT-based Applications
 \item 3 mini-demos of prototypical MMT-based applications 
  \lec{easy for attendants to understand, reprocude, modify}
   \begin{enumerate}
    \item Changing equality by adding arbitrary rewrite or computation rules
    \item Using the MMT query interfaces to build a browser-based editor
    \item Using MMT's export infrastructure to build an OpenMath Content Dictionary editor
   \end{enumerate}
 \end{enumerate}
\end{myframe}

\end{document}

